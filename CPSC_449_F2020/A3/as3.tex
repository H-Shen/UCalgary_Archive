\documentclass[11pt]{article}

\usepackage{amsmath,amsfonts,amssymb,amsthm}
\usepackage{mathtools}
\usepackage{fullpage}
\usepackage{tikz,graphicx}
\usetikzlibrary{automata, positioning, arrows, shapes}
\usepackage[margin=1in]{geometry}
\usepackage[utf8]{inputenc}
\usepackage[english]{babel}
\usepackage[shortlabels]{enumitem}

% Use listings to insert the haskell code
\usepackage{listings}
\usepackage{color}
\usepackage{xcolor}
\definecolor{dkgreen}{rgb}{0,0.6,0}
\definecolor{gray}{rgb}{0.5,0.5,0.5}
\definecolor{mauve}{rgb}{0.58,0,0.82}
\lstset{frame=tb,
     language=Haskell,
     aboveskip=3mm,
     belowskip=3mm,
     showstringspaces=false,
     columns=flexible,
     basicstyle = \ttfamily\small,
     numberstyle=\tiny\color{gray},
     keywordstyle=\color{blue},
     commentstyle=\color{dkgreen},
     stringstyle=\color{mauve},
     breaklines=true,
     breakatwhitespace=true,
     tabsize=3
}

\parindent 0pt
\parskip 3mm

\theoremstyle{definition}
\newtheorem*{solution}{Solution}

\begin{document}

\begin{center}
{\bf \Large CPSC 449 --- Principles of Programming Languages

Theory Components of Assignment 3}
\end{center}

\newenvironment{titlemize}[1]{%
  \paragraph{#1}
  \begin{itemize}}
  {\end{itemize}}
  
\hrule 	

\textbf{Name:} Haohu Shen \\
\textbf{Student ID:} 30063099 \\
\textbf{Tutorial:} 03 

\medskip \hrule

\begin{enumerate} \itemsep 20pt

\item[] \textbf{Problem 2}

A $polynomial\text{ }with\text{ }one\text{ }variable$ is represented using the following algebraic type.
\begin{lstlisting}[language=Haskell]
data Polynomial
= PConst Integer
| PVar
| PAdd Polynomial Polynomial
| PMul Polynomial Polynomial
\end{lstlisting}
The $degree$ of a $Polynomial$ can be obtained by the function $degree\text{ :: }Polynomial\text{ -> }Integer$, which is
defined by the following equations:
\begin{lstlisting}[language=Haskell]
degree :: Polynomial -> Integer
degree (PConst n) = 0                               -- (degree.1)
degree PVar = 1                                     -- (degree.2)
degree (PAdd p1 p2) = max (degree p1) (degree p2)   -- (degree.3)
degree (PMul p1 p2) = (degree p1) + (degree p2)     -- (degree.4)
\end{lstlisting}
The $first\text{ }derivative$ of a $Polynomial$ can be computed using the function $d\text{ :: }Polynomial\text{ -> }Polynomial$,
the definition of which is given by the following equations.
\begin{lstlisting}[language=Haskell]
d (PConst n) = PConst 0 -- (d.1)
d PVar = PConst 1 -- (d.2)
d (PAdd p1 p2) = PAdd (d p1) (d p2) -- (d.3)
d (PMul p1 p2) = PAdd (PMul p1 (d p2)) (PMul (d p1) p2) -- (d.4)
\end{lstlisting}
These four equations are the direct encoding of the well-known formulas in standard calculus textbooks.
\begin{align*}
\frac{dC}{dx} &= 0 \\
\frac{dx}{dx} &= 1 \\
\frac{d(u+v)}{dx} &= \frac{du}{dx} + \frac{dv}{dx} \\
\frac{d(uv)}{dx} &= u\frac{dv}{dx} + v\frac{du}{dx}
\end{align*}
Prove, by structural induction, that the following inequality holds for all finite $Polynomial$ $p$.
\begin{align*}
\text{degree p} &\geq \text{degree (d p)} \tag{1}
\end{align*}
\begin{enumerate}[(a)]
\item State the Principle of Structural Induction for $Polynomial$s.

\textbf{Solution.} From the description above we have the theorem $P(p)$ as
\begin{align*}
	\text{degree p} &\geq \text{degree (d p)}
\end{align*}
Thus to prove that $P(p)$ holds for all finite $Polynomial\text{ p}$, prove the following:
\begin{enumerate}
\item $P\text{(Pconst Integer)}$
\item $P\text{(PVar)}$
\item $P\text{(p1)} \land P\text{(p2)} \implies P\text{(PAdd p1 p2)}$
\item $P\text{(p1)} \land P\text{(p2)} \implies P\text{(PMul p1 p2)}$
\end{enumerate}
Therefore, the proof goals are:

\textbf{Proof Goals}
\begin{itemize}
  \item \textbf{Base Cases}
\begin{align*}
\text{degree (PConst n)} &\geq \text{degree (d (PConst n))} \tag{\text{base.1}} \\
\text{degree PVar} &\geq \text{degree (d PVar)} \tag{base.2}
\end{align*}
	\item \textbf{Induction Steps}
		\begin{itemize}
		\item \textbf{Assume:}
\begin{align*}
\text{degree p1} &\geq \text{degree (d p1)} \tag{\text{hyp.1}} \\
\text{degree p2} &\geq \text{degree (d p2)} \tag{\text{hyp.2}}
\end{align*}
		\item \textbf{Prove:}
\begin{align*}
\text{degree (PAdd p1 p2)} &\geq \text{degree (d (PAdd p1 p2))} \tag{\text{ind.1}} \\
\text{degree (PMul p1 p2)} &\geq \text{degree (d (PMul p1 p2))} \tag{\text{ind.2}}
\end{align*}
		\end{itemize}
\end{itemize}

\item Prove the base case(s).

Firstly, we prove the first base case:
\begin{itemize}
\item \textbf{Want:}
\begin{align*}
\text{degree (PConst n)} &\geq \text{degree (d (PConst n))}
\end{align*}
\item \textbf{Left-hand side:}
\begin{align*}
\text{degree (PConst n)} &= 0 \tag{\text{by degree.1}}
\end{align*}
\item \textbf{Right-hand side:}
\begin{align*}
\text{degree (d (PConst n))} &= \text{degree (PConst 0)} \tag{\text{by d.1}} \\
							 &= 0 \tag{\text{by degree.1}} \\
							 &\leq \text{L.H.S} \tag{\text{by arith.}}
\end{align*}
Thus it shows that the left-hand-side is greater than or equal to the right-hand-side, which completes the proof of the first base case.
\end{itemize}

Secondly, we prove the second base case:
\begin{itemize}
\item \textbf{Want:}
\begin{align*}
\text{degree PVar} &\geq \text{degree (d PVar)}
\end{align*}
\item \textbf{Left-hand side:}
\begin{align*}
\text{degree PVar} &= 1 \tag{\text{by degree.2}}
\end{align*}
\item \textbf{Right-hand side:}
\begin{align*}
\text{degree (d PVar)} &= \text{degree (PConst 1)} \tag{\text{by d.2}} \\
					   &= 0 \tag{\text{by degree.1}} \\
					   &\leq \text{L.H.S} \tag{\text{by arith.}}
\end{align*}
Thus it shows that the left-hand-side is greater than or equal to the right-hand-side, which completes the proof of the second base case.
\end{itemize}
Since all base cases are proved, we completed the proof of the base case of the statement.

\item Prove the induction step(s).

Firstly, we prove the first induction step:
\begin{itemize}
\item \textbf{Assume:}
\begin{align*}
\text{degree p1} &\geq \text{degree (d p1)} \tag{\text{hyp.1}} \\
\text{degree p2} &\geq \text{degree (d p2)} \tag{\text{hyp.2}}
\end{align*}
\item \textbf{Want:}
\begin{align*}
\text{degree (PAdd p1 p2)} &\geq \text{degree (d (PAdd p1 p2))}
\end{align*}
\item \textbf{Left-hand side:}
\begin{align*}
\text{degree (PAdd p1 p2)} &= \text{max (degree p1) (degree p2)} \tag{\text{by degree.3}}
\end{align*}
\item \textbf{Right-hand side:}
\begin{align*}
\text{degree (d (PAdd p1 p2))} &= \text{degree (PAdd (d p1) (d p2))} \tag{\text{by d.3}} \\
							   &= \text{max (degree (d p1)) (degree (d p2))} \tag{\text{by degree.3}}
\end{align*}
Thus, we can split the possible situations of the left-hand-side into 2 cases:
\begin{itemize}
\item \textbf{Case 1: }If $\text{degree p1} \geq \text{degree p2}$, then we continue the deduction in the left-hand-side 
and we have:
\begin{align*}
\text{L.H.S} &= \text{max (degree p1) (degree p2)} \\
			 &= \text{degree p1} \tag{\text{by defn. of max}} \\
			 &\geq \text{degree (d p1)} \tag{\text{by hyp.1}}
\end{align*} 
Also, we have
\begin{align*}
\text{L.H.S} &= \text{max (degree p1) (degree p2)} \\
			 &= \text{degree p1} \tag{\text{by defn. of max}} \\
			 &\geq \text{degree p2} \tag{\text{by cond. of the case}} \\
			 &\geq \text{degree (d p2)} \tag{\text{by hyp.2}}
\end{align*}
Since $\text{L.H.S} \geq \text{degree (d p1)}$ and $\text{L.H.S} \geq \text{degree (d p2)}$, we have
\begin{align*}
\text{L.H.S} &\geq \text{max (degree (d p1)) (degree (d p2))} \tag{\text{by defn. of max}}
\end{align*}
Since $\text{R.H.S} = \text{max (degree (d p1)) (degree (d p2))}$, we have $\text{L.H.S} \geq \text{R.H.S}$. Thus in this case, the left-hand-side is greater than or equal to the right-hand-side.

\item \textbf{Case 2: }If $\text{degree p2} \geq \text{degree p1}$, then we continue the deduction in the left-hand-side 
and we have:
\begin{align*}
\text{L.H.S} &= \text{max (degree p1) (degree p2)} \\
			 &= \text{degree p2} \tag{\text{by defn. of max}} \\
			 &\geq \text{degree (d p2)} \tag{\text{by hyp.2}}
\end{align*} 
Also, we have
\begin{align*}
\text{L.H.S} &= \text{max (degree p1) (degree p2)} \\
			 &= \text{degree p2} \tag{\text{by defn. of max}} \\
			 &\geq \text{degree p1} \tag{\text{by cond. of the case}} \\
			 &\geq \text{degree (d p1)} \tag{\text{by hyp.1}}
\end{align*}
Since $\text{L.H.S} \geq \text{degree (d p1)}$ and $\text{L.H.S} \geq \text{degree (d p2)}$, we have
\begin{align*}
\text{L.H.S} &\geq \text{max (degree (d p1)) (degree (d p2))} \tag{\text{by defn. of max}}
\end{align*}
Since $\text{R.H.S} = \text{max (degree (d p1)) (degree (d p2))}$, we have $\text{L.H.S} \geq \text{R.H.S}$. Thus in this case, the left-hand-side is greater than or equal to the right-hand-side.
\end{itemize}
Therefore, in both cases we have proved that the left-hand-side is greater than or equal to the right-hand-side,
on the assumption that the induction hypothesis holds, which completes the induction step.
\end{itemize}

Secondly, we prove the second induction step:
\begin{itemize}
\item \textbf{Assume:}
\begin{align*}
\text{degree p1} &\geq \text{degree (d p1)} \tag{\text{hyp.1}} \\
\text{degree p2} &\geq \text{degree (d p2)} \tag{\text{hyp.2}}
\end{align*}
\item \textbf{Want:}
\begin{align*}
\text{degree (PMul p1 p2)} &\geq \text{degree (d (PMul p1 p2))}
\end{align*}
\item \textbf{Left-hand side:}
\begin{align*}
\text{degree (PMul p1 p2)} &= \text{(degree p1)} + \text{(degree p2)} \tag{\text{by degree.4}}
\end{align*}
\item \textbf{Right-hand side:}
\begin{align*}
\text{degree (d (PMul p1 p2))} &= \text{degree (PAdd (PMul p1 (d p2)) (PMul (d p1) p2))} \tag{\text{by d.4}}
\end{align*}
Let $s1 = \text{PMul p1 (d p2)}$, $s2 = \text{PMul (d p1) p2}$, then we have
\begin{align*}
\text{R.H.S} &= \text{degree (PAdd s1 s2)} \tag{\text{by replacement}} \\
             &= \text{max (degree s1) (degree s2)} \tag{\text{by degree.3}}
\end{align*}
Since
\begin{align*}
\text{degree s1} &= \text{degree (PMul p1 (d p2))} \tag{\text{by replacement}} \\
				 &= \text{(degree p1) + (degree (d p2))} \tag{\text{by degree.4}} \\
				 &\leq \text{(degree p1) + (degree p2)} \tag{\text{by hyp.2}}
\end{align*}
Also
\begin{align*}
\text{degree s2} &= \text{degree (PMul (d p1) p2)} \tag{\text{by replacement}} \\
				 &= \text{(degree (d p1)) + (degree p2)} \tag{\text{by degree.4}} \\
				 &\leq \text{(degree p1) + (degree p2)} \tag{\text{by hyp.1}}
\end{align*}
Thus, we can split the possible situations of the right-hand-side into 2 cases:

\begin{itemize}
\item \textbf{Case 1: }If $\text{degree s1} \geq \text{degree s2}$, then we continue the deduction in the right-hand-side 
and have
\begin{align*}
\text{R.H.S} &= \text{max (degree s1) (degree s2)} \\
			 &= \text{degree s1} \tag{\text{by defn. of max}} \\ 
			 &\leq \text{(degree p1) + (degree p2)} \tag{\text{from the deduction in RHS}} \\
\end{align*}
Since $\text{L.H.S} = \text{(degree p1) + (degree p2)}$, we have $\text{L.H.S} \geq \text{R.H.S}$. Thus in this case, 
the left-hand-side is greater than or equal to the right-hand-side.

\item \textbf{Case 2: }If $\text{degree s2} \geq \text{degree s1}$, then we continue the deduction in the right-hand-side 
and have
\begin{align*}
\text{R.H.S} &= \text{max (degree s1) (degree s2)} \\
			 &= \text{degree s2} \tag{\text{by defn. of max}} \\ 
			 &\leq \text{(degree p1) + (degree p2)} \tag{\text{from the deduction in RHS}} \\
\end{align*}
Since $\text{L.H.S} = \text{(degree p1) + (degree p2)}$, we have $\text{L.H.S} \geq \text{R.H.S}$. Thus in this case, 
the left-hand-side is greater than or equal to the right-hand-side.
\end{itemize}
Therefore, in both cases we have proved that the left-hand-side is greater than or equal to the right-hand-side,
on the assumption that the induction hypothesis holds, which completes the induction step.
\end{itemize}
Since all induction steps are completed, we can say the proof itself is also completed. \qed
\end{enumerate}

\end{enumerate}

\end{document}
