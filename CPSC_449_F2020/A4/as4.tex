\documentclass[11pt]{article}

\usepackage{amsmath,amsfonts,amssymb,amsthm}
\usepackage{mathtools}
\usepackage{fullpage}
\usepackage{tikz,graphicx}
\usetikzlibrary{automata, positioning, arrows, shapes}
\usepackage[margin=1in]{geometry}
\usepackage[utf8]{inputenc}
\usepackage[english]{babel}
\usepackage[shortlabels]{enumitem}

% Use listings to insert the haskell code
\usepackage{listings}
\usepackage{color}
\usepackage{xcolor}
\definecolor{dkgreen}{rgb}{0,0.6,0}
\definecolor{gray}{rgb}{0.5,0.5,0.5}
\definecolor{mauve}{rgb}{0.58,0,0.82}
\lstset{frame=tb,
     language=Haskell,
     aboveskip=3mm,
     belowskip=3mm,
     showstringspaces=false,
     columns=flexible,
     basicstyle = \ttfamily\small,
     numberstyle=\tiny\color{gray},
     keywordstyle=\color{blue},
     commentstyle=\color{dkgreen},
     stringstyle=\color{mauve},
     breaklines=true,
     breakatwhitespace=true,
     tabsize=3
}

\parindent 0pt
\parskip 3mm

\theoremstyle{definition}
\newtheorem*{solution}{Solution}

\begin{document}

\begin{center}
{\bf \Large CPSC 449 --- Principles of Programming Languages

Theory Components of Assignment 4}
\end{center}

\newenvironment{titlemize}[1]{%
  \paragraph{#1}
  \begin{itemize}}
  {\end{itemize}}
  
\hrule 	

\textbf{Name:} Haohu Shen \\
\textbf{Student ID:} 30063099 \\
\textbf{Tutorial:} 03 

\medskip \hrule

\begin{enumerate} \itemsep 20pt

\item[] \textbf{Problem 2}

Use the following definition of the function $concat$:
\begin{verbatim}
concat = foldr (++) []
\end{verbatim}
You may also use the axiom (map++) on page 261 (under Exercise 11.31).
\begin{verbatim}
map f (ys ++ zs) = map f ys ++ map f zs
\end{verbatim}
Prove that for all finite lists $xs$, and functions $f$,
\begin{verbatim}
concat (map (map f) xs) = map f (concat xs)
\end{verbatim}

\textbf{Solution.} Before giving any proof, we list the definitions of the functions
$map$, $foldr$ and all extended properties we are allowed to use, notice that all contents
are referred from the textbook:
\begin{lstlisting}[language=Haskell]
concat = foldr (++) [] -- (concat.3)

-- map.1 and map.2 are in Page 217 of the textbook
map f [] = [] -- (map.1)
map f (x:xs) = f x : map f xs -- (map.2)
map f (ys ++ zs) = map f ys ++ map f zs -- (map++)

-- foldr.1 and foldr.2 are in Page 222 of the textbook
foldr f s [] = s -- (foldr.1)
foldr f s (x:xs) = f x (foldr f s xs) -- (foldr.2)
\end{lstlisting}

Now we prove the statement in the problem holds for all finite lists $xs$ and functions $f$ by structural induction:

\textbf{Proof Goals}
\begin{itemize}
  \item \textbf{Base Case}
  	\begin{verbatim}
  	concat (map (map f) []) = map f (concat [])                     (base)
	\end{verbatim}
	\item \textbf{Induction Step}
		\begin{itemize}
		\item \textbf{Assume:}
		\begin{verbatim}
		concat (map (map f) xs) = map f (concat xs)                     (hyp)
		\end{verbatim}
		\item \textbf{Prove:}
		\begin{verbatim}
		concat (map (map f) (x:xs)) = map f (concat (x:xs))            (ind)
		\end{verbatim}
		\end{itemize}
\end{itemize}

\textbf{Base Case}
\begin{itemize}
\item \textbf{Want:}
\begin{verbatim}
concat (map (map f) []) = map f (concat [])
\end{verbatim}
\item \textbf{Left-hand side:}
\begin{verbatim}
concat (map (map f) [])
= concat []                                                     by (map.1)
= foldr (++) [] []                                              by (concat.3)
= []                                                            by (foldr.1)
\end{verbatim}
\item \textbf{Right-hand side:}
\begin{verbatim}
map f (concat [])
= map f (foldr (++) [] [])                                      by (concat.3)
= map f []                                                      by (foldr.1)
= []                                                            by (map.1)
= L.H.S.
\end{verbatim}
Thus it shows that the two sides are the same, which completes the proof of the base case.
\end{itemize}

\textbf{Induction Step}
\begin{itemize}
\item \textbf{Assume:}
\begin{verbatim}
concat (map (map f) xs) = map f (concat xs)                         (hyp)
\end{verbatim}
\item \textbf{Want:}
\begin{verbatim}
concat (map (map f) (x:xs)) = map f (concat (x:xs))
\end{verbatim}
\item \textbf{Left-hand side:}
\begin{verbatim}
concat (map (map f) (x:xs))
= concat ((map f x) : map (map f) xs)                    by (map.2)
= foldr (++) [] ((map f x) : map (map f) xs)             by (concat.3)
= map f x ++ (foldr (++) [] (map (map f) xs))            by (foldr.2)
\end{verbatim}
Let
\begin{verbatim}
p = map (map f) xs
\end{verbatim}
Then we have
\begin{verbatim}
L.H.S
= map f x ++ (foldr (++) [] p)                         by replacement
= map f x ++ concat p                                  by (concat.3)
= map f x ++ concat (map (map f) xs)                   by replacement
= map f x ++ map f (concat xs)                         by (hyp)
\end{verbatim}
\item \textbf{Right-hand side:}
\begin{verbatim}
map f (concat (x:xs))
= map f (foldr (++) [] (x:xs))                         by (concat.3)
= map f (x ++ (foldr (++) [] xs))                      by (foldr.2)
= map f (x ++ concat xs)                               by (concat.3)
= map f x ++ map f (concat xs)                         by (map++)
= L.H.S.
\end{verbatim}
Since the final step makes the left- and right-hand sides equal, on the assumption that the induction hypothesis holds, This completes the induction step, and therefore the proof itself. \qed
\end{itemize}

\end{enumerate}

\end{document}
