\documentclass[11pt]{article}

\usepackage{amsmath,amsfonts,amssymb,amsthm}
\usepackage{mathtools}
\usepackage{fullpage}
\usepackage{tikz,graphicx}
\usetikzlibrary{automata, positioning, arrows, shapes}
\usepackage[margin=1in]{geometry}
\usepackage[utf8]{inputenc}
\usepackage[english]{babel}
\usepackage[shortlabels]{enumitem}

% Use listings to insert the haskell code
\usepackage{listings}
\usepackage{color}
\usepackage{xcolor}
\definecolor{dkgreen}{rgb}{0,0.6,0}
\definecolor{gray}{rgb}{0.5,0.5,0.5}
\definecolor{mauve}{rgb}{0.58,0,0.82}
\lstset{frame=tb,
     language=Haskell,
     aboveskip=3mm,
     belowskip=3mm,
     showstringspaces=false,
     columns=flexible,
     basicstyle = \ttfamily\small,
     numberstyle=\tiny\color{gray},
     keywordstyle=\color{blue},
     commentstyle=\color{dkgreen},
     stringstyle=\color{mauve},
     breaklines=true,
     breakatwhitespace=true,
     tabsize=3
}

\parindent 0pt
\parskip 3mm

\theoremstyle{definition}
\newtheorem*{solution}{Solution}

\begin{document}

\begin{center}
{\bf \Large CPSC 449 --- Principles of Programming Languages

Theory Components of Assignment 2}
\end{center}

\newenvironment{titlemize}[1]{%
  \paragraph{#1}
  \begin{itemize}}
  {\end{itemize}}
  
\hrule 	

\textbf{Name:} Haohu Shen \\
\textbf{Student ID:} 30063099 \\
\textbf{Tutorial:} 03 

\medskip \hrule

\begin{enumerate} \itemsep 20pt

\item[] \textbf{Problem 3}

\begin{enumerate}[(a)]
\item Prove that the implementation of merge sort in 2(c) always terminates.

\textbf{Solution.} In order to prove the implementation of merge sort in 2(c) always terminates, that is, the $mSort$ function terminates for
all inputs, we firstly prove all functions that are involved terminate for all inputs, that is, the $splitList$, 
$getEvenIndexItems$, $getOddIndexItems$, $mergeLists$ functions terminate for all inputs. Before giving any prove, the
definitions of all functions involved are listed below:

\begin{lstlisting}[language=Haskell]
mergeLists :: [Integer] -> [Integer] -> [Integer]
mergeLists xs [] = xs -- (mergeLists.1)
mergeLists [] ys = ys -- (mergeLists.2)
mergeLists (x : xs) (y : ys) -- (mergeLists.3)
	| x <= y = x : mergeLists xs (y : ys)
	| otherwise = y : mergeLists (x : xs) ys

getEvenIndexItems :: [Integer] -> [Integer]
getEvenIndexItems [] = [] -- (getEvenIndexItems.1)
getEvenIndexItems (x:_:xs) = x:getEvenIndexItems xs -- (getEvenIndexItems.2)
getEvenIndexItems (x:xs) = x:getEvenIndexItems xs -- (getEvenIndexItems.3)

getOddIndexItems :: [Integer] -> [Integer]
getOddIndexItems [] = [] -- (getOddIndexItems.1)
getOddIndexItems (_:x:xs) = x:getOddIndexItems xs -- (getOddIndexItems.2)
getOddIndexItems (_:xs) = getOddIndexItems xs -- (getOddIndexItems.3)

splitList :: [Integer] -> ([Integer], [Integer])
splitList [] = ([], []) -- (splitList.1)
splitList xs = (getEvenIndexItems xs, getOddIndexItems xs) -- (splitList.2)

mSort :: [Integer] -> [Integer]
mSort [] = [] -- (mSort.1)
mSort [x] = [x] -- (mSort.2)
mSort xs = mergeLists (mSort ys) (mSort zs) -- (mSort.3)
  where
    (ys, zs) = splitList xs
\end{lstlisting}

\begin{enumerate}[i.]

\item Firstly we prove that $getEvenIndexItems$ function terminates for all inputs, in order to do that, we define a 
function $rankGetEvenIndexItem$ that maps the input list to the length of the list, which is a non-negative integer.
Consider that the input list $xs$ has length of $n$, then:
\begin{lstlisting}[language=Haskell]
rankGetEvenIndexItem :: Integer -> Integer
rankGetEvenIndexItem 0 = 0 -- (rankE.1)
rankGetEvenIndexItem (1 + n) = 1 + rankGetEvenIndexItem n -- (rankE.2)
\end{lstlisting}

Thus, for the 2nd equation in the definition of $getEvenIndexItems$ such that it involves 1 recursive call, we prove 
that \\
$rankGetEvenIndexItem \text{(2 + n)} > rankGetEvenIndexItem \text{ n}$:
\begin{verbatim}
rankGetEvenIndexItem (2 + n)
= rankGetEvenIndexItem (1 + (1 + n))                           by arith.
= 1 + rankGetEvenIndexItem (1 + n)                             by (rankE.2)
= 1 + (1 + rankGetEvenIndexItem n)                             by (rankE.2)
= 2 + rankGetEvenIndexItem n                                   by arith.
> rankGetEvenIndexItem n                                       by arith.
\end{verbatim}
As required. \\
On the other hand, for the 3rd equation in the definition of $getEvenIndexItems$ such that it involves 1 recursive call, 
we prove that \\
$rankGetEvenIndexItem \text{ (1 + n)} > rankGetEvenIndexItem \text{ n}$:
\begin{verbatim}
rankGetEvenIndexItem (1 + n)
= 1 + rankGetEvenIndexItem n                                   by (rankE.2)
> rankGetEvenIndexItem n                                       by arith.
\end{verbatim}
As required. \\
Therefore, the function $rankGetEvenIndexItem$ maps the function $getEvenIndexItems$ to a natural number and there is a strict decrease of $rankGetEvenIndexItem$ 
when the recursion occurs, that is, we can say the function $getEvenIndexItems$ always terminates for all inputs.

\item Secondly we prove that $getOddIndexItems$ function terminates for all inputs, in order to do that, we define a 
function $rankGetOddIndexItem$ that maps the input list to the length of the list, which is a non-negative integer.
Consider that the input list $xs$ has length of $n$, then:
\begin{lstlisting}[language=Haskell]
rankGetOddIndexItem :: Integer -> Integer
rankGetOddIndexItem 0 = 0 -- (rankO.1)
rankGetOddIndexItem (1 + n) = 1 + rankGetOddIndexItem n -- (rankO.2)
\end{lstlisting}

Thus, for the 2nd equation in the definition of $getOddIndexItems$ such that it involves 1 recursive call, we prove 
that \\
$rankGetOddIndexItem \text{ (2 + n)} > rankGetOddIndexItem \text{ n}$:

\begin{verbatim}
rankGetOddIndexItem (2 + n)
rankGetOddIndexItem (1 + (1 + n))                             by arith.
= 1 + rankGetOddIndexItem (1 + n)                             by (rankO.2)
= 1 + (1 + rankGetOddIndexItem n)                             by (rankO.2)
= 2 + rankGetOddIndexItem n                                   by arith.
> rankGetOddIndexItem n                                       by arith.
\end{verbatim}
As required. \\
On the other hand, for the 3rd equation in the definition of $getOddIndexItems$ such that it involves 1 recursive call, 
we prove that \\
$rankGetOddIndexItem \text{ (1 + n)} > rankGetOddIndexItem \text{ n}$:

\begin{verbatim}
rankGetOddIndexItem (1 + n)
= 1 + rankGetOddIndexItem xs                                  by (rankO.2)
> rankGetOddIndexItem xs                                      by arith.
\end{verbatim}
As required. \\
Therefore, the function $rankGetOddIndexItem$ maps the function $getOddIndexItems$ to a natural number and there is a strict decrease of $rankGetOddIndexItem$ 
when the recursion occurs, that is, we can say the function $getOddIndexItems$ always terminates for all inputs.

\item Thirdly we prove that $splitList$ function terminates for all inputs, in order to do that, we define a 
function $rankSplitList$ that maps the input list to the length of the list, which is a non-negative integer.
Consider that the input list $xs$ has length of $n$, then:

\begin{lstlisting}[language=Haskell]
rankSplitList :: Integer -> Integer
rankSplitList 0 = 0	-- (rankS.1)
rankSplitList n = rankGetEvenIndexItem n + rankGetOddIndexItem n -- (rankS.2)
\end{lstlisting}

Thus, for the 2nd equation in the definition of $splitList$ such that it involves 1 recursive call, the length of the 
input list is at least 1, thus\\
\begin{itemize}
\item \textbf{Case 1:} If the length of input is at least 2, then we prove that we prove
that \\
$rankSplitList \text{ (2 + n)} > rankSplitList \text{ n}$.

\begin{verbatim}
rankSplitList (2+n)
= rankGetEvenIndexItem(2+n)+rankGetOddIndexItem(2+n)        by (rankS.2)
= rankGetEvenIndexItem(1+(1+n))+rankGetOddIndexItem(2+n)    by arith.
= 1+rankGetEvenIndexItem(1+n)+rankGetOddIndexItem(2+n)      by (rankE.2)
= 1+rankGetEvenIndexItem(1+n)+rankGetOddIndexItem(1+(1+n))  by arith.
= 1+rankGetEvenIndexItem(1+n)+1+rankGetOddIndexItem(1+n)    by (rankO.2)
= 2+rankGetEvenIndexItem(1+n)+rankGetOddIndexItem(1+n)      by arith.
= 2+(1+rankGetEvenIndexItem n)+rankGetOddIndexItem(1+n)     by (rankE.2)
= 2+(1+rankGetEvenIndexItem n)+(1+rankGetOddIndexItem n)    by (rankO.2)
= 4+(rankGetEvenIndexItem n + rankGetOddIndexItem n)        by arith.
= 4+rankSplitList n                                         by (rankS.2)
> rankSplitList n                                           by arith.
\end{verbatim}

\item \textbf{Case 2:} Otherwise, the length of input is 1, then we prove that \\
$rankSplitList \text{ (1+n)} > rankSplitList \text{ n}$.

\begin{verbatim}
rankSplitList (1+n)
= rankGetEvenIndexItem (1+n) + rankGetOddIndexItem (1+n)       by (rankS.2)
= (1 + rankGetEvenIndexItem n) + rankGetOddIndexItem (1+n)     by (rankE.2)
= (1 + rankGetEvenIndexItem n) + (1 + rankGetOddIndexItem n)   by (rankO.2)
= 2 + (rankGetEvenIndexItem n + rankGetOddIndexItem n)         by arith.
= 2 + rankSplitList n                                          by (rankS.2)
> rankSplitList n                                              by arith.
\end{verbatim}
\end{itemize}
Therefore, the function $rankSplitList$ maps the function $splitList$ to a natural number and there is a strict decrease of $rankSplitList$ 
when the recursion occurs, that is, we can say the function $rankSplitList$ always terminates for all inputs.

\item Next we prove that $mergeLists$ function terminates for all inputs, in order to do that, we define a 
function $rankMergeLists$ that maps the input of two lists of integers to a non-negative number, since the recursival call
only exists in the 3rd definition of $mergeList$, we consider its input $(x:xs)$ and $(y:ys)$ as arguments:
\begin{itemize}
	\item Suppose the length of $xs$ is $n$ and the length of $ys$ is $m$.
	\item Then the $rankMergeLists$ of the inputs is:
	\begin{verbatim}
	rankMergeLists (1+n) (1+m)
	= 1 + rankMergeLists n (1+m)
	> rankMergeLists n (1+m)
	\end{verbatim}
	or
	\begin{verbatim}
	rankMergeLists (1+n) (1+m)
	= 1 + rankMergeLists (1+n) m
	> rankMergeLists (1+n) m
	\end{verbatim}
	Such that $\text{ n (1+m)}$ or $\text{ (1+n) m}$ are the arguments of $rankMergeLists$ in the recursive call on
	the RHS. 
\end{itemize}
Since there is a strict decrease of value of $rankMergeLists$ when recursion occurs, we can say the function 
$mergeLists$ terminates for all inputs.

\item Finally we prove that the function $mSort$ terminates for all inputs, in order to do that, we define a function
$rank$ that maps the input list to the length of the list, which is a non-negative integer.
Consider that the input list $xs$ has length of $n$, then:

\begin{lstlisting}[language=Haskell]
rank :: Integer -> Integer
rank 0 = 0	-- (rankS.1)
rank 1 = 1 -- (rankS.2)
rank n = rank a + rank b -- (rankS.3)
\end{lstlisting}
Such that $a$ is the length of $getEvenIndexItems \textbf{ xs}$, $b$ is the length of \\
$getOddIndexItems \textbf{ xs}$.

The 3rd equation involves 2 recursive calls. Since the length of $xs$ is $n$ and $n >= 2$ when the equation is called, we have:
\begin{itemize}
	\item For the first recursive call, the argument of the recursive call is
	$a$. Since $a$ is the length of $getEvenIndexItems \textbf{ xs}$ such that $n >= 2$, thus $a < n$. Moreover, $rank$
	maps the input list to the length of the list, thus $rank \text{ n}= \text{n}$, $rank \text{ a}= \text{a}$, thus
	$rank \text{ n} > rank \text{ a}$, which indicates that $rank \text{ a}$ is strictly smaller than the input rank.
	\item For the second recursive call, the argument of the recursive call is
	$b$. Since $b$ is the length of $getOddIndexItems \textbf{ xs}$ such that $n >= 2$, thus $b < n$. Moreover, $rank$
	maps the input list to the length of the list, thus $rank \text{ n}= \text{n}$, $rank \text{ b}= \text{b}$, thus
	$rank \text{ n} > rank \text{ b}$, which indicates that $rank \text{ b}$ is strictly smaller than the input rank.
\end{itemize}

Since both ranks of argument decrease strictly as the recursion occurs, we can say that the function $mSort$ is guaranteed
to terminate for all inputs. \qed

\end{enumerate}

\item Consider the following function.
\begin{lstlisting}[language=Haskell]
mystery :: [[Integer]] -> Integer
mystery [] = 0
mystery ((x:xs):ys) = x + mystery (xs:ys)
mystery ([]:ys) = mystery ys
\end{lstlisting}
\begin{enumerate}[i.]
\item Give a conjecture of what the $mystery$ function returns.

\textbf{Solution.} The $mystery$ function takes a list of lists of integers and returns the summation of the sum of all sub-lists.
\item Prove that the $mystery$ function terminates for all inputs.

\textbf{Proof.} In order to prove the function terminates for all inputs, we define a function $rank$ that maps the input
list to the sum of total length of sub-lists and the number of sub-lists inside the input list. That is:
\begin{lstlisting}[language=Haskell]
rank :: [[Integer]] -> Integer
rank [] = 0 -- (rank.1)
rank ((x : xs) : ys) = 1 + rank (xs : ys) -- (rank.2)
rank ([] : ys) = 1 + rank ys -- (rank.3)
\end{lstlisting}   
Thus, for the 2nd equation in the definition of $mystery$ such that it involves 1 recursive call, we prove that $rank ((x:xs):ys) > rank(xs:ys)$:
\begin{verbatim}
rank ((x:xs):ys)
= 1 + rank (xs : ys)                                      by (rank.2)
> rank (xs : ys)                                          by arith.
\end{verbatim}
On the other hand, for the 3rd equation in the definition of $mystery$ such that it involves 1 recursive call, we prove that $rank ([\text{ }]:ys) > rank(ys)$:
\begin{verbatim}
rank ([]:ys)
= 1 + rank ys                                             by (rank.3)
> rank ys                                                 by arith.
\end{verbatim}
Therefore, the function $rank$ maps the function $mystery$ to a natural number and there is a strict decrease of $rank$ when the recursion occurs, that is, we can say the function $mystery$ always terminates for all inputs. 
\end{enumerate}
\end{enumerate}

\item[] \textbf{Problem 4}
Prove, by structural induction, for all finite lists $xs$, we have: 
\begin{lstlisting}[language=Haskell]
length xs = length (reverse xs)		-- (*)
\end{lstlisting}
You may assume the following equations for $reverse$ and $length$:
\begin{lstlisting}[language=Haskell]
reverse [] = [] -- (reverse.1)
reverse (x : xs) = (reverse xs) ++ [x] -- (reverse.2)

length [] = 0	-- (length.1)
length (x : xs) = 1 + (length xs)	-- (length.2)
length (xs ++ ys) = (length xs) + (length ys)	-- (length.3)
\end{lstlisting}
Adhere to the following steps in your proof.
\begin{enumerate}[(a)]
\item State the two proof goals (i.e., base case and induction step).

\textbf{Proof Goals}
\begin{itemize}
  \item \textbf{Base Case}
  	\begin{verbatim}
length [] = length (reverse [])                              (base)
	\end{verbatim}
	\item \textbf{Induction Step}
		\begin{itemize}
		\item \textbf{Assume:}
		\begin{verbatim}
length xs = length (reverse xs)                          	 (hyp)
		\end{verbatim}
		\item \textbf{Prove:}
		\begin{verbatim}
length (x:xs) = length (reverse (x:xs))                    (ind)
		\end{verbatim}
		\end{itemize}
\end{itemize}

\item Prove the base case.
\begin{itemize}
\item \textbf{Want:}
  	\begin{verbatim}
length [] = length (reverse [])
	\end{verbatim}
\item \textbf{Left-hand side:}
	\begin{verbatim}
	length []
= 0                                                    by (length.1)
	\end{verbatim}
\item \textbf{Right-hand side:}
	\begin{verbatim}
	length (reverse [])
= length []                                            by (reverse.1)
= 0                                                    by (length.1)
= L.H.S.  
	\end{verbatim}
	Thus it shows that the two sides are the same, which completes the proof of the base case.
\end{itemize}
\item Prove the induction step.
\begin{itemize}
	\item \textbf{Assume:}
		  \begin{verbatim}
			length xs = length (reverse xs)                                (hyp)
		\end{verbatim}
	\item \textbf{Want:}
		\begin{verbatim}
			length (x:xs) = length (reverse (x:xs))
		\end{verbatim}
	\item \textbf{Left-hand side:}
		\begin{verbatim}
		length (x:xs)
		= 1 + (length xs)                                       by (length.2)
		= 1 + length xs
		= 1 + length (reverse xs)                                    by (hyp)
		\end{verbatim}
	\item \textbf{Right-hand side:}
		\begin{verbatim}
		length (reverse (x:xs))
		= length ((reverse xs) ++ [x])                          by (reverse.2)
		= (length (reverse xs)) + (length [x])                  by (length.3)
		= length (reverse xs) + length [x]
		= length (reverse xs) + length (x : [])                 by defn. of [x]
		= length (reverse xs) + (1 + (length []))               by (length.2)
		= length (reverse xs) + (1 + length [])
		= length (reverse xs) + (1 + 0)                         by (length.1)
		= length (reverse xs) + 1
		= 1 + length (reverse xs)					                                   by arith.
		= L.H.S.
		\end{verbatim}
		Since the final step makes the left- and right-hand sides equal, on the assumption that the induction hypothesis holds, This completes the induction step, and therefore the proof itself. \qed
	\end{itemize}
\end{enumerate}

\end{enumerate}

\end{document}
